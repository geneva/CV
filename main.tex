%% start of file `main.tex'.
%% Copyright 2014 Francois Mouton (moutonf@gmail.com).
%
% This template is adapted from the work performed by Xavier Danaux (xdanaux@gmail.com).
% This template further extends the functionality by integrating the moderntimeline package.
% This template also includes custom Biblatex style to print bibliography items with the moderntimeline package.
%
% This work may be distributed and/or modified under the
% conditions of the LaTeX Project Public License version 1.3c,
% available at http://www.latex-project.org/lppl/.


\documentclass[11pt,letterpaper,roman]{moderncv/moderncv}        % possible options include font size ('10pt', '11pt' and '12pt'), paper size ('a4paper', 'letterpaper', 'a5paper', 'legalpaper', 'executivepaper' and 'landscape') and font family ('sans' and 'roman')

% moderncv themes
\moderncvstyle{classic}                             % Only the 'classic' style is fully functional with the modifications made. The other options, 'casual' (default), 'oldstyle' and 'banking' has minor typesetting problems with the current modifications.
\moderncvcolor{blue}                               % color options 'blue' (default), 'orange', 'green', 'red', 'purple', 'grey' and 'black'
%\renewcommand{\familydefault}{\sfdefault}         % to set the default font; use '\sfdefault' for the default sans serif font, '\rmdefault' for the default roman one, or any tex font name

% character encoding
\usepackage[utf8]{inputenc}                       % if you are not using xelatex ou lualatex, replace by the encoding you are using

% adjust the page margins
\usepackage[scale=0.75]{geometry}
%\setlength{\hintscolumnwidth}{3cm}                % if you want to change the width of the column of the timeline
%\setlength{\makecvtitlenamewidth}{10cm}           % for the 'classic' style, if you want to force the width allocated to your name and avoid line breaks. Be careful though, the length is normally calculated to avoid any overlap with your personal info; use this at your own typographical risks.

%-------------------Inlcuding pdfpages package-------------------------------------------------------------

\usepackage{pdfpages/pdfpages}

%-------------------Including moderntimeline package-------------------------------------------------------

\usepackage{moderntimeline/moderntimeline}

\tlmaxdates{2001}{2017}                             % Set the scale of the timeline. \tlmaxdates{startDate}{endDate}

%-------------------Including xpatch package---------------------------------------------------------------

\usepackage{xpatch/xpatch}

%-------------------Including Biblatex package-------------------------------------------------------------

\usepackage[url=false,
    backend=bibtex,                                  % This can be set to either biber or bibtex. If references are missing just change back and forth between biber and bibtex..
    style=authoryear,
    doi=false,  
    isbn=false,
    backref=false,
    dashed=false,                                   % Do not add a dash out authors for subsequent articles with the same authors.
    maxnames=99,                                    % Amount of authors to include before abbreviating.
    sorting=ydnt]{biblatex}                         % Sorting in reverse order

\addbibresource{cvreferences.bib}                   % Include your bibtex file here. Format: fileName.bib
\renewbibmacro{in:}{}
\renewbibmacro{pp.}{}
\DeclareFieldFormat{pages}{#1}
\input{biblatex_modifications/standard_modification.tex}        % Modifying the default standard.tex style of Biblatex. This modification is performed to include the moderntimeline package.

%-------------------Defining a CV Reference column style and a CV reference entry block-------------------

% Adapted from the solution provided in: http://tex.stackexchange.com/questions/34881/references-section-in-a-cv
% usage: \cvreference{name}{address line 1}{address line 2}{address line 3}{address line 4}{e-mail address}{phone number}{mobile phone number}
% Everything but the name is optional
% If \addresssymbol, \emailsymbol or \phonesymbol are specified, they will be used.
% (Per default, \addresssymbol isn't specified, the other two are specified.)
% If you don't like the symbols, remove them from the following code, including the tilde ~ (e.g. \phonesymbol~).

\newcommand{\cvreferencecolumn}[2]{%
  \cvitem[0.75em]{}{%
    \begin{minipage}[t]{\listdoubleitemmaincolumnwidth}#1\end{minipage}%
    \hfill%
    \begin{minipage}[t]{\listdoubleitemmaincolumnwidth}#2\end{minipage}%
    }%
}

\newcommand{\cvreference}[8]{%
    \textbf{#1}\newline% Name
    \ifthenelse{\equal{#2}{}}{}{\addresssymbol~#2\newline}%
    \ifthenelse{\equal{#3}{}}{}{#3\newline}%
    \ifthenelse{\equal{#4}{}}{}{#4\newline}%
    \ifthenelse{\equal{#5}{}}{}{#5\newline}%
    \ifthenelse{\equal{#6}{}}{}{\emailsymbol~\texttt{\href{mailto:#6}{\nolinkurl{#6}}}\newline}%
    \ifthenelse{\equal{#7}{}}{}{\phonesymbol~#7\newline}
    \ifthenelse{\equal{#8}{}}{}{\mobilephonesymbol~#8}}
    
\newcommand{\genevabold}[1] {\textbf{#1}}

%-------------------Personal Data for CV title-----------------------------------------------------------
% Example:
\name{Anthony J}{Geneva}
\title{Curriculum Vitae}                               % optional, remove / comment the line if not wanted
\address{Havard University}{Museum of Comparative Zoology}{Cambridge, MA 02138}% optional, remove / comment the line if not wanted; the "postcode city" and and "country" arguments can be omitted or provided empty
\phone[mobile]{+1~(215)~219~8709}                   % optional, remove / comment the line if not wanted
%\phone[fixed]{+2~(345)~678~901}                    % optional, remove / comment the line if not wanted
%\phone[fax]{+3~(456)~789~012}                      % optional, remove / comment the line if not wanted
\email{geneva@crescatscientia.com}                               % optional, remove / comment the line if not wanted
\homepage{CrescatScientia.com}                         % optional, remove / comment the line if not wanted
\extrainfo{github.com/geneva}    
%\extrainfo{additional information}                 % optional, remove / comment the line if not wanted
%\photo[64pt][0.4pt]{images/picture}                       % optional, remove / comment the line if not wanted; '64pt' is the height the picture must be resized to, 0.4pt is the thickness of the frame around it (put it to 0pt for no frame) and 'picture' is the name of the picture file stored
%\quote{Some quote}                                 % optional, remove / comment the line if not wanted

%-------------------------------------------------------------------------------------------------------
%   Content
%-------------------------------------------------------------------------------------------------------
\begin{document}

%-------------------Resume------------------------------------------------------------------------------

\makecvtitle

%-------------------Education Section-------------------------------------------------------------------

\section{Education}

% For a date range: (To indicate 'up to present', set EndYear to 0)
% Format:  \tlcventry{StartYear}{EndYear}{Degree}{Institution}{City}{\textit{Grade}}{Description}  % Arguments 3 (Degree) to 6 (Grade) can be left empty. 
% Example: \tlcventry{2012}{0}{BSc Computer Science}{University of MyCity}{MyCity}{}{Also completed several random courses}

\tlcventry{2015}{0}{Post-Doctoral Fellow}{Harvard University}{}{}{supervisor: Jonathan Losos}

\tldatecventry{2015}{PhD Candidate}{University of Rochester}{}{}{supervisors: Daniel Garrigan \& Richard Glor}

% For a single year:
% Format:  \tldatecventry{StartYear}{Degree}{Institution}{City}{\textit{Grade}}{Description}
% Example: \tldatecventry{2008}{Senior Certificate}{High School MyCity}{MyCity}{\textit{80\%}}{Passed with distinction}

\tldatecventry{2007}{MS Biology}{Villanova University}{}{}{supervisors: Todd Jackman \& Aaron Bauer}

\tldatecventry{2001}{BS Biology (Genetics)}{Penn State University}{}{}{minors: Information Systems \& Statistical Analysis, Information Science \& Technology}


%-------------------PhD Thesis Section------------------------------------------------------------------

%\section{PhD thesis}

% Format:  \cvitem{Section Name}{Description}
% Example: \cvitem{title}{\emph{The title of my PhD goes here}}
% Example: \cvitem{supervisors}{My supervisors' names go here}
% Example: \cvitem{description}{Short thesis abstract}

%\cvitem{title}{Genomics of Speciation in \emph{Anolis} Lizards}
%\cvitem{supervisors}{Daniel Garrigan \& Richard Glor}
%\cvitem{description}{Short thesis abstract}

%-------------------Masters Thesis Section--------------------------------------------------------------

%\section{Masters thesis}

% Format:  \cvitem{Section Name}{Description}
% Example: \cvitem{title}{\emph{The title of my Masters goes here}}
% Example: \cvitem{supervisors}{My supervisors' names go here}
% Example: \cvitem{description}{Short thesis abstract}

%\cvitem{title}{A Phylogeographic Analysis of \emph{Bavayia cf. crassicollis}}
%\cvitem{supervisors}{Todd Jackman \& Aaron Bauer}
%\cvitem{description}{Short thesis abstract}

%-------------------Achievements Section----------------------------------------------------------------

%\section{Achievements}

% Format:  \cvlistitem{Achievement}
% Example: \cvlistitem{Received best student award}
% Example: \cvlistitem{Another achievement. This achievement is particularly long and therefore normally spans over several lines. Did you notice the indentation when the line wraps?}

%\cvlistitem{Received best student award}
%\cvlistitem{Another achievement. This achievement is particularly long and therefore normally spans over several lines. Did you notice the indentation when the line wraps?}



\section{Professional Experience}

% Format: \tlcventry{StartYear}{EndYear}{Job title}{Employer}{City}{Country (optional)}{General description no longer than 1--2 lines.\newline{}%
% Example:
 \tlcventry{2005}{2009}{Manager - Laboratory of Molecular Systematics \& Ecology}{The Academy of Natural Sciences}{Philadelphia}{}{}%Did system administrative work.\newline{}%
% Main Duties:%
%  \begin{itemize}%
%      \item Administrate the servers;
%      \item Administrate employee computers 
%         \begin{itemize}%
%              \item All employee's computers had to be up to date;
%          \end{itemize}
%      \item Did some more administrating
%   \end{itemize}}
 \tlcventry{2002}{2005}{Graduate Research Fellow}{Villanova Univeristy}{}{}{}
%  \tlcventry{2002}{0}{Independent Contractor}{Florida Park Service}{}{}{}{}
%    \begin{itemize}
%      \item Constructed fire weather database
%   \end{itemize}
   \tlcventry{2001}{2002}{AmeriCorps Ecology Intern}{Florida Park Service}{}{}{}
%   \tlcventry{1998}{2001}{Laboratory Assistant}{Dr. Blair Hedges}{Penn State University}{}{}

%-------------------Skills Matrix Section----------------------------------------------------------------

%\section{Skills}

% For items with categories: 
% Format:  \cvdoubleitem{Category}{List of skills}{Category Name}{List of skills}
% Note: It looks better if the category is bold with \textbf{}
% Example:
% \subsection{Development}
% \cvdoubleitem{\textbf{Languages}}{C\#, C\+\+, Java}{\textbf{Databases}}{MSSQL, MySQL}
%
% For a bullet list without categories:
% Format:  \cvlistdoubleitem{Skill 1}{Skill 2}
% Example: 
% \subsection{Development}
% \cvlistdoubleitem{C\#, Java, Ruby}{MSSQL, MySQL}
% \cvlistdoubleitem{Photoshop}{Windows, Linux. In the single column list, this item is particularly long to wrap over several lines.}

%\subsection{Development}
%\cvdoubleitem{\textbf{Languages}}{R, Bash, Perl, C}{\textbf{HPC}}{SLURM, TORQUE}


%-------------------Publications Section----------------------------------------------------------------
% The cvitem commands needs to be altered to correctly print all publications with the moderntime package.
% The cvitem command is edited to remove all forced punctuation within the command.
% All the typesetting of the text is handled by the modified Biblatex style.

\input{cvitem_modifications/cvitem_modified}        % Removing forced punctuation from cvitem

\nocite{*}                                          % Print all publications.

% Format:  \printbibliography[type=Biblatex type,title={Title of publication}]
% Example: \printbibliography[type=article,title={Journal Publications}]
% Example: \printbibliography[type=inproceedings,title={Conference Publications}]
% Example: \printbibliography[type=thesis,title={Thesis}]

%\printbibliography[type=misc,title={Preprints}]
\printbibliography[type=article,title={Journal Publications}]


\section{Research Grants \& Fellowships}

\tldatecventry{2015}{Doctoral Dissertation Improvement Grant}{}{}{}{\$19,995}

\tlcventry{2011}{2014}{Ernst Caspari Fellowship}{}{}{}{\$6,000}	

\tlcventry{2010}{2014}{Google Earth Education Initiative}{awarded as software licenses}{}{}{\$5,985 per year, renewed annually}

\tldatecventry{2011}{Sigma Xi Grants-in-Aid}{Speciation in Bahamian Trunk Anoles}{}{}{\$400} 

\tldatecventry{2011}{Sproull Dissertation Travel Grant}{}{}{}{\$800}

\tlcventry{2009}{2011}{Robert L. And Mary L. Sproull University Fellowship}{}{}{}{\$44,000}

\tldatecventry{2007}{Laboratory for Molecular Systematics and Ecology Small Grants}{Getting to the bottom of molecular discord in New Caledonian \emph{Bavayia} geckos}{}{}{\$6,166}


\section{Awards \& Recognition}

\tldatecventry{2011}{SSAR Best Poster Award - Evolution, Genetics, \& Systematics}{}{}{}{}
\tldatecventry{2011}{SSAR Graduate Travel Award}{}{}{}{}	
\tldatecventry{2009}{Accepted to Bodega Applied Phylogenetics Workshop}{}{Bodega Bay, CA}{}{}
\tldatecventry{2009}{Accepted to Estimating Species Tree Workshop}{}{Ann Arbor, MI}{}{}
\tldatecventry{2008}{Sigma Xi Annual Meeting Superior Presentation Award}{}{}{}{}
\tldatecventry{2004}{ASIH Graduate Travel Award}{}{}{}{}	
\tldatecventry{2004}{Villanova Summer Research Fellowship}{}{}{}{}
\tldatecventry{2004}{Villanova Graduate Research Fellowship}{}{}{}{}
\tldatecventry{2003}{Villanova Summer Research Fellowship}{}{}{}{}	

\section{Software Contributions}
\cvlistitem{\texttt{spectrum2stats}: An R package to calculate population genetic summary statistics from site frequency spectra \emph{github.com/geneva/spectrum2stats}}

\cvlistitem{\texttt{POPBAMTools}: An R package for reading and annotating output from POPBAM \emph{github.com/geneva/POPBAMTools}}

\cvlistitem{\texttt{AFLPTools}: Tools for the curation and analysis of AFLP data \emph{github.com/geneva/AFLPTools}}

\cvlistitem{\texttt{R We There Yet?}: An R package for the analysis of Markov chain Monte Carlo convergence in Bayesian phylogenetic inference (contributor) \emph{github.com/danlwarren/RWTY}}

\cvlistitem{\texttt{ngscmd}: A C program to manipulate next-generation sequence data files \mbox{(contributor)} \emph{github.com/dgarriga/ngscmd}}


%\section{Teaching Experience}
%\tldatecventry{2014}{Guest Lecturer}{}{Population Genetics and Phylogeography}
%\tldatecventry{2014}{Teaching Assistant}{American Genetic Association Non-model Genomics Workshop}{Ithaca, NY}{}{}

%\tldatecventry{2011}{Graduate Coordinator}{Freshman Level Advanced Biology}{Rochester}{}{}

%\tldatecventry{2010}{Teaching Assistant}{Freshman Level Advanced Biology}{Rochester}{}{}

%\tldatecventry{2003}{Teaching Assistant}{Sophomore Level General Biology}{Villanova}{}{}

%\tldatecventry{2003}{Teaching Assistant}{Senior Level Evolution}{Villanova}{}{}

%\tldatecventry{2002}{Teaching Assistant}{Freshman Level General Biology}{Villanova}{}{}

%\tldatecventry{2002}{Teaching Assistant}{Senior Level Developmental Biology}{Villanova}{}{}

%Broader Impacts
%Outreach
%2008 – present 	Scientific Advisor: Geckos: Tails to Toepads traveling exhibit
%Exhibit on the science of new species discovery, over 1 million visitors at six Natural History/science museums
	
%Blogging
%Contributor to Anole Annals – Clearinghouse of information on Anolis research
%anoleannals.org > 70,000 total hits, ~700 unique visits/day

%Student Mentoring  - University of Rochester Undergraduates
%Shannon Keating (thesis) – molecular population genetics, phylogenetics
%Sabina Noll - bioinformatics
%Jared Hilton – multilocus phylogenetics
%Christine Rienhart – specimen data management and analysis
%Erin Hatch – experimental hybridizations
%Frank Chang – molecular methods
%Benjamin Desch (thesis) – molecular population genetics
%Hillary Goldman – molecular methods

%Student Mentoring – NSF-REU (Research Experience for Undergraduates)
%2008	Susan Tsang, Project Title: Ecology and Genetics of a Rare Fish
%2006	Patrick Videau, Project Title: Developing DNA extraction protocols from Museum Preserved Specimens


\printbibliography[type=thesis,title={Thesis}]
\printbibliography[type=inproceedings,title={Conference Presentations}]
%\printbibliography[type=poster,title={Conference Presentations}]

\input{cvitem_modifications/cvitem_moderncvclassic} % Reverting changes to cvitem.





%-------------------References Section------------------------------------------------------------------

%\section{Referees}

% Format:  \cvreferencecolumn{\cvreference{Name Surname}{Position}{Department}{Company}{City}{Email}{Home Phone}{Cell Phone}}{\cvreference{Name Surname}{Position}{Department}{Company}{City}{Email}{Home Phone}{Cell Phone}}
% Example: 
% \subsection{Simple Solutions}
% \cvreferencecolumn{\cvreference{John Doe}{Developer}{HR}{Simple Solutions}{MyCity}{john@email.com}{+12 (34) 567 8901}{+23 (45) 678 9012}}{\cvreference{Jane Doe}{Accountant}{HR}{Simple Solutions}{MyCity}{jane@email.com}{+34 (56) 789 0123}{+45 (67) 890 1234}}
% 
% \cvreferencecolumn{\cvreference{Alice Doe}{Manager}{HR}{Monster Inc}{ThatCity}{alice@email.com}{+12 (34) 567 8901}{+23 (45) 678 9012}}{}
%\subsection{Thesis Advisers}
%\cvreferencecolumn{\cvreference{Daniel Garrigan}{Assistant Professor}{}{University of Rochester}{Rochester, NY 14627}{dgarri@rochester.edu}{(124) 567 8901}{}}
%{\cvreference{Richard E Glor}{Curator \& Associate Professor}{}{University of Kansas}{Lawrence, KS}{glor@ku.edu}{(123) 789 0123}{}}


%\subsection{Additional References}
%\cvreferencecolumn{\cvreference{Daven Presgraves}{Professor}{}{University of Rochester}{Rochester, NY}{dvpg@rochester.edu}{+12 (34) 567 8901}{}}
%{\cvreference{Someone else}{Professor}{}{University of X}{XX, XX}{XX@XX.edu}{(156) 789 0123}{}}


\clearpage

%-------------------Appendix----------------------------------------------------------------------------
% This section is added to append any additional documents to the cv.
% The appended documents are added to the table of contents for easier navigation of the document.
% Usage: (section)
% \phantomsection
% \addcontentsline{toc}{section}{title}
% 
% Format: (subsection)
% \phantomsection\addcontentsline{toc}{subsection}{title}
% \includepdf[pages=-]{appendix/filename.pdf}
%
% Example:
% \phantomsection
% \addcontentsline{toc}{section}{Certificates}
%
% \phantomsection
% \addcontentsline{toc}{subsection}{Landscape}
% \includepdf[pages=-]{appendix/CertificateLandscape.pdf}
%
% \phantomsection
% \addcontentsline{toc}{subsection}{Portrait}
% \includepdf[pages=-]{appendix/CertificatePortrait.pdf}


%-------------------Cover letter------------------------------------------------------------------------

%\input{coverletter.tex}                             % Include cover letter from coverletter.tex

%-------------------Document End------------------------------------------------------------------------

\end{document}

%% end of file `main.tex'.
